%\documentclass{llncs}
\documentclass[letterpaper,11pt]{llncs}
%\documentclass{article} % llncs

\usepackage{usenix}
\usepackage{url}
\usepackage{amsmath}
\usepackage{epsfig}
\usepackage{epsf}
\usepackage{listings}

%\setlength{\textwidth}{6in}
%\setlength{\textheight}{8.4in}
%\setlength{\topmargin}{.5cm}
%\setlength{\oddsidemargin}{1cm}
%\setlength{\evensidemargin}{1cm}

\begin{document}

\title{The Future of HTTP and Anonymity on the Internet}

% XXX: This is broken:
\author{Georg Koppen \\ The Tor Project, Inc \\ georg@torproject.org}
\author{Mike Perry \\ The Tor Project, Inc \\ mikeperry@torproject.org}

%\institute{The Internet}

\maketitle
\pagestyle{plain}

\begin{abstract}

The Tor Project has a keen interest in the development of future standards
involving the HTTP application layer and associated transport layers. At
minimum, we seek to ensure that all future HTTP standards remain compatible
with the Tor Network, avoid introducing new third party tracking and
linkability vectors, and minimize client fingerprintability. We also have a
strong interest in the development of enhancements and/or extensions that
protect the confidentiality and integrity of HTTP traffic, as well as provide
resistance to traffic fingerprinting and general traffic analysis. In fact, we
are presently researching these areas.

\end{abstract}

\section{Introduction}

The Tor Project is a United States 501(c)(3) non-profit dedicated to providing
technology, research, and education to support online privacy, anonymity, and
censorship circumvention. Our primary software products are the Tor network
software, and the Tor Browser, which is based on Firefox. The Tor Project is
actively collaborating with Mozilla to ensure that its modifications to
Firefox are merged with the official Firefox distribution, with the long-term
goal of providing an optional Tor-enabled mode of operation for native Firefox
users.

In this position paper, we describe the concerns and interests of the Tor
Project with respect to future HTTP standardization. These concerns and
interests span six areas: identifier linkability, connection usage linkability,
fingerprinting linkability, traffic confidentiality and integrity, traffic
fingerprinting and traffic analysis, and Tor network compatibility.

Each of these areas is communicated in a separate section of this position
paper. We have also performed a preliminary review of HTTP/2 with respect to
these areas, and have noted our comments in each section. We will be
performing a more in-depth review of HTTP/2 for client fingerprinting and
other tracking issues in the coming months as we modify the Firefox
implementation for deployment in Tor Browser.

\section{Identifier Linkability Concerns}

Identifier linkability is the ability to use any form of browser state, cache,
data storage, or identifier to track (or link) a user between two otherwise
independent actions. For the purpose of this position paper, we are
specifically concerned with any stateful information residing in the browser's
HTTP layer that persists beyond the duration or scope of a single connection.

For background, the Tor Project has designed Tor Browser with two main
properties for limiting identifier-based tracking: First Party Isolation, and
Long Term Unlinkability.
% FIXME: cite design doc

First Party Isolation is the property that a user's actions at one
top-level URL bar domain must not be correlated or linked to their actions on a
different top-level URL bar domain. We maintain this property through a number
of patches and modifications to various aspects of browser functionality and
state keeping.
% FIXME: Cite design doc

Long Term Unlinkability is the property that a user's future activity must not
be linked or correlated to any prior activity after that user's explicit request
to sever this link. Tor Browser provides Long Term Unlinkability by allowing
the user to clear all browser tracking data in a single click (called "New
Identity"). Our long-term goal is to allow users to define their relationship
with individual first parties, and alter that relationship by resetting or
blocking the associated tracking data on a per-site basis.
% FIXME: Cite design doc

\subsection{Identifier Linkability in HTTP/2}

The Tor Project is still in the process of evaluating the stateful nature of
HTTP/2 connections and their associated streams and settings. It is likely
that we will be able to isolate the usage of HTTP/2 connection state in a
manner similar to the way we currently isolate HTTP/1.1 connection state.
However, it is not clear yet at this point how complicated this isolation will
be.

\subsection{Avoiding Future Identifier Linkability}

We feel that it is very important that mechanisms for identifier usage,
storage, and connection-related state keeping be cleanly abstracted and
narrowly scoped within the HTTP protocol. However, we also recognize that to a
large degree identifier storage and the resulting linkability is primarily an
implementation detail, and not specific to the protocol itself.

Identifier linkability will become a problem if instances arise where the
server is allowed to specify a setting or configuration property for a client
that must persist beyond the duration of the session. In these cases, care
must be taken to ensure that this data is cleared or isolated upon entry to
private browsing mode, or when the user attempts to clear their private data.
In the case of Tor Browser, we will most likely clear this state immediately
upon connection close.

\section{Connection Usage Linkability Concerns}

Connection usage linkability arises from the use of the same underlying
transport stream for requests that would otherwise be independent due to the
first party isolation of their associated identifiers and browser state.

Tor Browser currently enforces connection usage unlinkability at the HTTP
layer, by creating independent HTTP connections for third party hosts that
are sourced from different first party domains.

\subsection{Connection Usage Linkability with HTTP/2}

The heavy use of connection multiplexing in HTTP/2 may present additional
complexities for ensuring that requests are isolated. Unfortunately, unlike
identifier usage, connection usage linkability is encouraged by the
HTTP/2 specification in Section 9.1 (in the form of specifying that clients
SHOULD NOT open more than one connection to a given host and port).
% FIXME: Cite HTTP2 S 9.1

\subsection{Avoiding Future Connection Usage Linkability}

In the future, connection usage linkability may become a problem if the notion
of a connection becomes disassociated from the application layer, and instead
is enforced through a collection of identifiers or stateful behavior in the
browser. This may tend to encourage implementations that make it difficult to
decouple the notion of a connection from the notion of a destination address.

Connection (and even identifier) linkability could similarly arise if
implementations were required to remember which endpoints supported which HTTP
versions, to avoid wasting round trips determining this information in-band.

Even these concerns are technically implementation issues, but consideration
should be taken to ensure that the specification does not encourage
implementations to bake in deep assumptions about providing only a single
connection instance per site, as well as the need to remember a site's
capabilities for long periods of time.

\section{Fingerprinting Linkability Concerns}

User agent fingerprinting arises from four sources: end-user configuration
details, device and hardware characteristics, operating system vendor and
version differences, and browser vendor and version differences.

The Tor Project is primarily concerned with minimizing the ability of websites
to obtain or infer end user configuration details and device characteristics.
We concern ourselves with operating system fingerprinting only to the point of
removing ways of detecting a specific operating system version. We make no
attempt to address fingerprinting due to browser vendor and version
differences. % FIXME: cite fingerprinting doc

Under this model, it is unlikely that very many fingerprinting vectors that
concern us will arise in the HTTP layer. However, the possibility for end user
configuration details to leak into behaviors of the HTTP layer is still a
possibility.

\subsection{Fingerprinting Linkability in HTTP/2}

The Tor Project is still in the process of evaluating client
fingerprintability in HTTP/2. The largest potential source of fingerprinting
appears to be in the SETTINGS frame. If clients choose setting values
depending on end-user configuration, hardware capabilities, or operating
system version, we may alter our implementation's behavior accordingly to
remove this logic.

\subsection{Avoiding Future Fingerprinting Linkability}

It is conceivable that more fingerprinting vectors could arise in future,
especially if more flow control and stream multiplexing decisions are
delegated to the client, and depend on things like available system memory,
available CPU cores, or other system details. Care should be taken to avoid
these situations, but we also expect them to be unlikely.

\section{Traffic Confidentiality and Integrity Concerns}

The Tor Project is very interested in any efforts to improve the
confidentiality and integrity of the session layer of HTTP/3. 

In particular, we are strong advocates for mandatory authenticated encryption
of HTTP/3 connections.  The availability of entry-level authentication through
the Let's Encrypt Project should eliminate the remaining barriers to requiring
authenticated encryption, as opposed to deploying opportunistic mechanisms.
% FIXME: Cite Let's Encrypt

We are also interested in efforts to encrypt the ClientHello and ServerHello
messages using an initial ephemeral handshake, as described in the Encrypted
TLS Handshake proposal. If SNI, ALPN, and the ServerHello can be encrypted
using an ephemeral key exchange that is authenticated later in the handshake,
the adversary loses a great deal of information about the user's intended
destination site. When large scale CDNs and multi-site load balancers are
involved, the ultimate destination would be impossible to determine with this
type of handshake in place. This will aid in defenses against traffic
fingerprinting and traffic analysis, which we describe detail in the next
section.
% FIXME: Cite https://tools.ietf.org/html/draft-ray-tls-encrypted-handshake-00

\section{Traffic Fingerprinting and Traffic Analysis Concerns}

Website Traffic Fingerprinting is the process of using machine learning to
classify web page visits based on their encrypted traffic patterns. It is most
effective when exact request and response lengths are visible, and when the
classification domain is limited by knowledge of the specific site being
visited.

In the case of Tor, this attack is most commonly considered with respect to
the client's connection to their Guard node (the entry into the Tor network).
There, Tor's fixed 512 byte packet size, link encryption, and stream
multiplexing go a long way to impede this attack for minimal overhead. The
fixed 512 byte packet size helps to obscure some amount of request and
response length information. Tor's link encryption also conceals the
destination website from the Guard node observer, which reduces classifier
accuracy and capabilities by increasing the size of the classification domain.

There was some initial controversy in the research literature as to the exact
degree to which the classification domain size, the base rate fallacy, and
other machine learning issues applied to website traffic fingerprinting of Tor
traffic, but after publicly requesting that these effects be studied in closer
detail, recent results have confirmed and quantified the benefits conferred by
Tor's unique link encryption.

Tor's link properties are by no means a complete defense, but they show that
there is room to develop defenses that specifically aim to increase the size
of the classification domain and associated base rate. In fact, it is our
belief that minimal padding and clever use of request and response framing
will increase the false positive rate enough to frustrate these attacks. For
this reason, we have been encouraging continued study of low-overhead defenses
against traffic fingerprinting. 
% FIXME: Cite blog post and Mjaurez's paper
% FIXME: Cite design doc's website traffic fingerprinting section

With the aid of an encrypted TLS handshake, we are hopeful that these defenses
will also be applicable to non-Tor TLS sessions as well. In addition to
protecting the communications of non-Tor users from traffic fingerprinting,
the application of these defenses to the HTTP TLS layer will serve to increase
the difficulty of end-to-end correlation and general traffic analysis of Tor
exit node traffic as well.

\subsection{Traffic Analysis Issues with HTTP/2}

In our preliminary investigation of HTTP/2, we discovered that certain aspects
of the protocol may aid certain types of traffic analysis attacks.

In particular, the PING and SETTINGS frames are acknowledged immediately by
the client, which might give servers the ability to collect information about a
client's location and/or routing via timing side-channels. They also allow the
server to introduce an active traffic pattern that can be used for end-to-end
correlation or confirmation.

In Tor Browser, we will likely introduce delay or jitter before responding to
these requests, and close the connection after receiving some rate of
unsolicited PING or SETTINGS updates. However, lack of explicit guidance in
the specification about this issue raises concerns about what frequencies of
these frames are likely to represent normal server behavior in the wild due to
overly-aggressive HTTP/2 implementations, as opposed to actual attacks.

It is true that there are other mechanisms that an attacker could use for the
same purpose (such as redirects or Javascript), but these mechanisms can
either be disabled by the user, discovered by UI indicators, or otherwise
mitigated by Tor Browser.

\subsection{Future Traffic Analysis Resistance Enhancements for HTTP/3}

In terms of assisting traffic analysis defenses, we would like to see
capabilities for larger amounts of per-frame padding, and more fine-grained
client-side control over frame sizes. Unfortunately the 256 bytes of padding
provided by HTTP/2 is likely to be inconsequential when combined with the
minimum frame size the client can request (16 kilobytes).

In combination with researchers at the University of Leuven, the Tor Project
has also developed a protocol and prototype implementation for communicating
statistical schedules for asynchronous padding from Tor clients to Tor relays.
The research community is currently in the process of evaluating the efficacy
of this protocol against traffic fingerprinting and other traffic analysis
attacks.
% FIXME: Cite padding idea and wfpadtools

Pending the results of this analysis, these padding commands could form the
basis of new HTTP/3 frame commands for communicating more sophisticated (yet
still traffic-bounded) padding schedules to HTTP/3 servers.


\section{Tor Network Compatibility Concerns}

Our final area of concern is continued compatibility of the Tor network with
future versions of the HTTP protocol. It is our understanding that there is a
desire for future versions of HTTP to move to a UDP transport layer so that
reliability, congestion control, and client mobility will be more directly
under control of the client user agent.

At present, the Tor Network is only capable of carrying TCP traffic. While it
will be possible to support the transit of UDP datagrams using our existing
TCP overlay network without significant anonymity risks within a year's time
or sooner, it is unlikely that this level of support will be sufficient to
warrant the use of a finely-tuned UDP version of HTTP rather than a TCP
variant.

Long term, our goal is to transition the entire Tor network to our own
datagram protocol with custom congestion and flow control to better support
both native datagram transport and end-to-end flow control. However,
additional research is still needed to examine the anonymity implications
associated with this transition. Our present estimate is that a full network
transition to UDP is at least five years away.
% FIXME: Site Murdoch's UDP study

We are also concerned that even after a full network transition to a datagram
transport, it is likely that the congestion, flow, and reliability control of
a UDP version of HTTP may still end up performing poorly over higher-latency
overlay networks such as ours.

For these reasons, we are especially interested in ensuring that overlay
networks are taken into account in the design of any UDP-based future versions
of HTTP, and also prefer to retain the ability to use future HTTP versions
over TCP, should the UDP implementations prove sub-optimal for our use case.



\bibliographystyle{plain} \bibliography{HTTP3}

\clearpage
\appendix

\end{document}
