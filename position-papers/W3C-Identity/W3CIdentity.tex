%\documentclass{llncs}
\documentclass[letterpaper,11pt]{llncs}
%\documentclass{article} % llncs

\usepackage{usenix}
\usepackage{url}
\usepackage{amsmath}
\usepackage{epsfig}
\usepackage{epsf}
\usepackage{listings}

%\setlength{\textwidth}{6in}
%\setlength{\textheight}{8.4in}
%\setlength{\topmargin}{.5cm}
%\setlength{\oddsidemargin}{1cm}
%\setlength{\evensidemargin}{1cm}

\begin{document}

\title{Bridging the Disconnect Between Web Identity and User Perception}

\author{Mike Perry \\ The Internet \\ mikeperry@torproject.org}

%\institute{The Internet}

\maketitle
\pagestyle{plain}

\begin{abstract}

There is a huge disconnect between how users perceive their online presence
and the reality of their relationship with the websites they visit. This
position paper explores this disconnect and provides some recommendations for
making the technical reality of the web match user perception, through both
technical improvements as well as user interface cues. By looking at all of
the elements of tracking as though they collectively comprise "User Identity",
we can make better decisions about improvements to both the technical and the
interface aspects of authentication and privacy.

\end{abstract}

\section{Introduction}

The prevailing revenue model of the web is an appealing one. Web users receive
unfettered, frictionless access to an extensive variety of information sources
in exchange for viewing advertising. This advertising is more valuable if each
advertisement is more relevant to the current activity, and if possible, more
relevant to the current user.

The cost of this is that user privacy on the web is a nightmare. There is
ubiquitous tracking, unseen partnership agreements and data exchange, and
surreptitious attempts to uncover users' identities against their will and
without their knowledge. This is not just happening in the dark, unseemly
corners of the web. It is happening everywhere\cite{facebook-like}.

The problem is that the revenue model of the web has incentivized companies to
find ways to continue to track users against their will, even if those users
are attempting to protect themselves through currently available methods.
Starting with the infamous "Flash cookies", we have progressed through a
seemingly endless arms race of secondary identifiers and tracking information:
visited history, cache, font and system data, desktop resolution, keystroke
timing, and so on and so forth\cite{wsj-fingerprinting}.

These efforts have lead to an even wider disconnect between a user's
perception of their privacy and the reality of their privacy. Users simply
can't keep up with the ways they are being tracked.

When users are being coerced into ceding data about themselves without clear
understanding or consent (and in fact, in many cases despite their explicit
attempts to decline to consent), serious moral issues begin to arise.

To understand and evaluate potential solutions and improvements to this status
quo, we must explore the disconnect between user experience and the way the
web actually functions with respect to tracking and identity.

% FIXME: Do we need this paragraph?
%To this end, the rest of this document is structured as follows: First, we
%examine user identity on the web, comparing the average user's perspective to
%what actually is happening technically behind the scenes, and noting the major
%disconnects. We then examine solutions attempting to bridge this disconnect
%from two different directions.

We only consider implementations that involve privacy-by-design.
Privacy-by-policy approaches such as Do Not Track will not be discussed.

\section{User Identity on the Web}

To properly examine this privacy problem, we must probe into the details of
both what a User's perception of their identity is, as well as the technical
realities of what goes into web authentication and tracking.

\subsection{User Perception of Identity}

Instinctively, users define their privacy in terms of their identity, in terms
of how they have interacted with a site in order to inform it of who they are.
Typically, the user's perception of their identity on the web is usually a direct
function of the mechanisms used for strong authentication for particular sites.

For example, users expect that logging in to Facebook creates a relationship
in their browsers when facebook.com is present in the URL bar, but they are
likely not aware that this also extends to their activity on other, arbitrary
sites that happen to include "Like this on Facebook" buttons or
Facebook-sourced advertising content.

Many, if not most, users expect that when they log out of a site their
relationship ends and that any associated tracking should be over. Even
users who are aware of cookies can be prone to believing that clearing the
cookies and private browsing data related to a particular site is sufficient
to end their relationship with that site.

Neither of these beliefs has any relation to reality.

\subsection{Technical Reality of Identity}

The technical reality of the web today is that users are usually wrong about
their authentication status with respect to a particular site, and are almost
always oblivious to the relationship between content elements of arbitrary
pages. The default experience is such that all of this data exchange is
concealed from the user.

So then what is identity? In terms of authentication, it would at first appear
to be cookies, HTTP Auth tokens, and client TLS certificates. However, even this
begins to break down. High-security websites are already using fingerprinting
as an auxiliary second factor of authentication\cite{security-fingerprinting},
and online data aggregators utilize everything they can to build complete
portraits of users' identities\cite{tracking-identity}.

Identity then is a superset of all the authentication tokens used by the
browser. It is the ability to link a user's activity in one instance to their
activity in another instance, be it across time, or even on a single page
due to multiple content origins.

\subsection{Identity as Linkability}

When expanded to cover all items that enable or substantially contribute to
Linkability, a lot more components of the browser are now in scope. We will
briefly enumerate these components.

First, the obvious properties are found in the state of the browser: cookies,
DOM storage, cache, cryptographic tokens and cryptographic state, and
location. These are what technical people tend to think of first when it comes
to private browsing and identity, but they are not the whole story.

Next, we have long-term properties of the browser itself. These include the
User Agent String, the list of installed plugins, rendering capabilities,
window decoration size, and browser widget size.

Then, we have properties of the computer. These include desktop size, IP
address, clock offset and timezone, and installed fonts.

Finally, linkability also includes the properties of the multi-origin model of
the web that allow tracking due to partnerships. These include the implicit
cookie transmission model, and also explicit click referral and data exchange
partnerships.

\subsection{Developing a Threat Model}

Unfortunately, just about every browser property and functionality is a
potential fingerprinting target. In order to properly address the network
adversary on a technical level, we need a metric to measure linkability of the
various browser properties that extend beyond any stored origin-related state.

The Panopticlick project by the EFF provides us with exactly this
metric\cite{panopticlick}. The researchers conducted a survey of volunteers
who were asked to visit an experiment page that harvested many of the above
components. They then computed the Shannon Entropy of the resulting
distribution of each of several key attributes to determine how many bits of
identifying information each attribute provided.

While not perfect\footnotemark, this metric allows us to prioritize effort at
components that have the most potential for linkability.

\footnotetext{In particular, the test does not take in all aspects of
resolution information. It did not calculate the size of widgets, window
decoration, or toolbar size. We believe this may add high amounts of entropy
to the screen field. It also did not measure clock offset and other time-based
fingerprints. Furthermore, as new browser features are added, this experiment
should be repeated to include them.}

This metric also indicates that it is beneficial to standardize on
implementations of fingerprinting resistance where possible. More
implementations using the same defenses means more users with similar
fingerprints, which means less entropy in the metric.

\section{Matching User Perception with Reality}

When the concept of user identity is expanded to cover all aspects of
linkability, addressing the problem of the disconnect between user perception
and reality becomes clearer. For users to have privacy, and for private
browsing modes to function, the relationship between a user and a site must be
understood by that user.

It is apparent that the user experiences disconnect with the technical
realities of the web on two major fronts: the average user does not grasp the
privacy implications of the multi-origin model, nor are they given a clear
concept of identity to grasp the privacy implications of the union of the
trackable components of their browsers.

We will now examine examples of attempts at reducing this disconnect on each
of these two fronts.

Note that identity-based approaches and the origin-based approaches are
orthogonal. They may be combined, or used independently.

\subsection{Origin-Based Approaches}

Origin-based approaches seek to improve the technical behavior of the browser
to make linkability less implicit and more consent-driven. In short, these
approaches seek to make the web behave more like users currently assume it
behaves by anchoring browser state to top-level origins as opposed to
associating it with arbitrary content elements.

The earliest relevant example of this work is SafeCache\cite{safecache}.
SafeCache seeks to reduce the ability for 3rd party content elements to use
the cache to store identifiers. It does this by limiting the scope of the
cache to the origin in the url bar. This has the effect that commonly sourced
content elements are fetched and cached repeatedly, but this is the desired
property. Each of these prevalent content elements can be crafted to include
unique identifiers for each user, tracking users who attempt to avoid tracking
by clearing cookies.

Mozilla has a wonderful example of an origin-based improvement written by Dan
Witte and buried on their wiki\cite{thirdparty}. It describes a new dual-keyed
origin for cookies, so that cookies would only be transmitted if they matched
both the top level origin and the third party origin involved in their
creation. This approach would go a long way towards preventing implicit
tracking across multiple websites.

Similarly, one could imagine this two-level origin isolation being deployed to
improve similar issues with DOM Storage and cryptographic tokens.

Making the origin model for browser identifiers more closely match the user
activity and user expectation has other advantages as well. With a clear
distinction between 3rd party and top-level cookies, the privacy settings
window could have a user-intuitive way of representing the user's relationship
with different origins, perhaps by using only the favicon of that top level
origin to represent all of the browser state accumulated by that origin. The
user could delete the entire set of browser state (cookies, cache, storage,
cryptographic tokens) associated with a site simply by removing its favicon
from their privacy info panel.

The problem with origin-based approaches is that individually, they do not
fully address the entire linkability problem unless the same restriction is
applied uniformly to all aspects of stored browser state, and all other
linkability issues are dealt with. Behind-the-scenes partnerships can easily
allow companies to continue to link users to their identities through any
aspect of browser state that is not properly compartmentalized to the top
level origin and bound to the same rules.

% FIXME: Wording..
However, linkability based on browser properties is amenable to this
model. In particular, one can imagine per-origin plugin permissions,
per-origin limits on the number of fonts that can be used, and randomized
window-specific time offsets.

So, while these approaches are in fact useful for bringing the technical
realities of the web closer to what the user assumes is happening, they must
be deployed uniformly, with a consistent top-level origin restriction model.
This may take significant coordination and standardization efforts.

\subsection{Identity-Based Approaches}

We will now discuss what we call the identity-based approaches to privacy.
These approaches, whether explicitly or implicitly, all model the user's web
identity as the entirety of the user's state for all origins.

The key advantage of identity-based approaches is that they can be simpler
than origin-based approaches when used to improve the privacy problem on their
own.

While the earliest example of an identity-based approach is our own work on
Torbutton\cite{torbutton}, Torbutton deserves poor marks for both simplicity
and usability\cite{not-to-toggle}. Torbutton attempts to isolate the user's
non-Tor activity from their Tor activity, effectively providing the user with
a blank slate for their Tor activity, but optionally allowing them to toggle
between these two identities.

Firefox Private Browsing Mode is similar, in that it allows users to switch
between their normal browsing and a "private" clean slate.

% FIXME: This paragraph can go if we need space:
Both Firefox PBM and Torbutton suffer from usability issues, primarily because
this concept of separate browsing identities is not properly conveyed to the
user. In Firefox's case, this usability issue is apparent through the quantity
of mode error observed in the review of Private Browsing Modes by Dan Boneh et
al\cite{private-browsing}. In Torbutton's case, the issues appear more severe.
We've informally observed that users have tremendous difficulties remembering
which tabs were Tor-related and which were non-Tor related, and we've also
observed issues with mode error.

Both of these approaches are exceedingly complex: they deal with every aspect
of browser state individually. This development effort however does enable
Firefox and Torbutton to provide the user with great fine-grained control.

Google Chrome's Incognito Mode comes the closest to conveying this idea of
"Incognito identity" to the user, and the implementation is also simpler as a
result. The Incognito Mode window is a separate, stylized window that clearly
conveys an alternate identity is in use for this window, which can be used
concurrent to the non-private identity. This appears to lead to less mode
error (where the user forgets their private browsing state) compared to other
browsers.

% FIXME: This paragraph can go if we need space:
The implementation of Incognito is as a virtualized in-memory profile, which
allows them to achieve protection against history storage issues with minimal
effort. It also allows them to tweak browser properties and permissions
specifically for this profile.

The Mozilla Weave project appears to be proposing an identity-based method of
managing, syncing, and storing authentication tokens, and also has use cases
described for multiple users of a single browser\cite{weave-manager}. It is
the closest idea on paper to what we envision as the way to bridge user
assumptions with reality.

We believe that the user interface of the browser should convey a sense of
persistent identity prominently to the user in the form of a visual cue. This
cue can either be an abstract image, graphic or theme (such as the user's
choice of Firefox Persona\cite{firefox-personas}), or it can be a text area
with the user's current favored pseudonym. This idea of identity should then
be integrated with the browsing experience. Users should be able to click a
button to get a clean slate for a new identity, and should be able to log in
and out of of password-protected stored encrypted identities, which would
contain the entire state of the browser. This is the direction the Tor Project
intends to head in with the Tor Browser Bundle\cite{not-to-toggle}.

To this user, the Private Browsing Mode would be no more than a special case
of this identity UI - a special identity that they can trust not to store
browsing history information to disk. Such a UI also more explicitly captures
what is going on with respect to the user's relationship to the web.

However, all current private browsing modes fall short of protecting against a
network adversary and fail to deal with linkability against a network
adversary\cite{private-browsing}, claiming that it is outside their threat
model\footnotemark. If the user is given a new identity that is still linkable
to the previous one due to shortcomings of the browser, this approach has
failed as a privacy measure.

\footnotetext{The primary reason given to abstain from addressing the network
adversary is IP address linkability. However, we believe this to be a red
herring. Users are quite capable of using alternate Internet connections, and
it is common practice for ISPs in many parts of the world to rotate user IP
addresses daily, to discourage servers and to impede the spread of malware.
This is especially true of cellular IP networks.}

Linkability solutions within the identity framework would be similar to the
origin-based solutions, except they would be properties of the entire browser
or browser profile, and would be obfuscated only once per identity switch.

% FIXME: Elaborate?

\section{Conclusions}

There is a demand for private browsing, and we believe that solid private
browsing modes can be created. In order to do this, we need solid analysis of
the threat models involved, and we need standardization for many aspects of
defense.

However, there is currently a huge disconnect between user privacy and
identity due to both the multi-origin nature of the web, and the failure of
browsers to adequately convey a sense of identity to the user. It is possible
to bridge this disconnect both by addressing the issues with the multi-origin
model, as well as providing the user with an explicit representation of their
web identity, and with control over this identity.

% XXX: The dangers of adblockers and filters + the long-term imperative of
% improving privacy for the continued use of the advertising revenue model.

\bibliographystyle{plain} \bibliography{W3CIdentity}

\clearpage
\appendix

\end{document}
